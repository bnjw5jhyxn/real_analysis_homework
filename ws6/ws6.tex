\def\abs#1{\vert#1\vert}
\centerline{Steve Hsu\hfill workshop 6}
\item{2.} a.

Let $A$ be a set of real numbers.
By definition, $(A^\circ)^c$ is the set of real numbers $x$
that are not in $A^\circ$, that is,
$(A^\circ)^c$ is the set of real numbers $x$ such that
there is no $\epsilon > 0$ satisfying $V_\epsilon(x) \subset A$.
In other words, $(A^\circ)^c$ is the set of real numbers $x$
such that for all $\epsilon > 0$, $V_\epsilon(x) \not\subset A$.
Also by definition, $\overline{A^c}$ is the set of real numbers $x$
such that there is a sequence $(a_n) \subset A^c$
that converges to $x$.

To show that $(A^\circ)^c \subset \overline{A^c}$,
let $x$ be an arbitrary element of $(A^\circ)^c$, that is,
assume that for all $\epsilon > 0$, $V_\epsilon(x) \not\subset A$.
For each natural number $n$, we have that $V_{(1/n)}(x) \not\subset A$,
that is, there is some real number outside of $A$ in $V_{(1/n)}(x)$.
Let $a_n$ be this real number.
Notice that for all natural numbers $n$, $a_n \notin A$, that is,
$(a_n) \subset A^c$.
We will show that $(a_n)$ converges to $x$.
Given $\epsilon > 0$ take $N$ such that $1/N < \epsilon$.
For all $n \ge N$, since $1/n \le 1/N < \epsilon$, we have that
$a_n \in V_{(1/n)}(x) \subset V_{(1/N)}(x)$, that is,
$\abs{a_n - x} \le 1/N < \epsilon$.
Therefore, there is a sequence $(a_n) \subset A^c$ that converges to $x$,
so $x$ is an element of $\overline{A^c}$.

To show that $\overline{A^c} \subset (A^\circ)^c$,
let $x$ be an arbitrary element of $\overline{A^c}$, that is,
assume that there is a sequence $(a_n) \subset A^c$ converging to $x$.
Given $\epsilon > 0$, since $(a_n)$ converges to $x$,
we can take $n$ such that $\abs{a_n - x} < \epsilon$.
By definition, $a_n \in V_\epsilon(x)$ and $a_n \in A^c$.
Since $a_n \in V_\epsilon(x)$ and $a_n \notin A$,
we have that $V_\epsilon(x) \not\subset A$.
Since for all $\epsilon > 0$, $V_\epsilon(x) \not\subset A$,
$x$ is an element of $(A^\circ)^c$, as desired.
\medskip
Let $B = A^c$.
Since $\overline{B^c} = (B^\circ)^c$ by part (a), we have that
$\overline A^c = \overline{(A^c)^c}^c = \overline{B^c}^c =
((B^\circ)^c)^c = B^\circ = (A^c)^\circ$, as desired.
\bigskip
\item{} b.

Let $B = (0,1) \cup (1,2)$.
Clearly, $B^\circ = (0,1) \cup (1,2)$, while
$\overline B^\circ = (0,2)^\circ = (0,2) \ne (0,1) \cup (1,2)$.
\medskip
Let $C = \{0\}$.
Clearly, $\overline C = \{0\}$, while
$\overline{C^\circ} = \overline \emptyset = \emptyset \ne \{0\}$.
\bye
