\centerline{Steve Hsu\hfill workshop 2}
\item{a.} i. true

By the definition of $f^{-1}$,
$f^{-1}(A \cup B) = \{x \in X : f(x) \in A \cup B\}$.
By the definition of set union, this set is equal to
$\{x \in X : f(x) \in A \lor f(x) \in B\}$.
Clearly, this set is
$\{x \in X : f(x) \in A\} \cup \{x \in X : f(x) \in B\}$,
which by definition of $f^{-1}$ is $f^{-1}(A) \cup f^{-1}(B)$.
\bigskip
\item{} ii. true

By the definition of $f^{-1}$,
$f^{-1}(A \cap B) = \{x \in X : f(x) \in A \cap B\}$.
By the definition of set intersection, this set is equal to
$\{x \in X : f(x) \in A \land f(x) \in B\}$.
Clearly, this set is
$\{x \in X : f(x) \in A\} \cap \{x \in X : f(x) \in B\}$,
which by definition of $f^{-1}$ is $f^{-1}(A) \cap f^{-1}(B)$.
\bigskip
\item{b.} i. true

First we will show that $f(A \cup B) \subset f(A) \cup f(B)$.
In other words, we want to show that
if $y \in f(A \cup B)$, then $y \in f(A) \cup f(B)$.

Assume that $y \in f(A \cup B)$.
We want to show that $y \in f(A) \cup f(B)$.
Since $y \in f(A \cup B)$, $y = f(x)$ for some $x \in A \cup B$.
By the definition of set union, we have two cases:
first, $x \in A$; and second, $x \in B$.
In the first case, since $x \in A$, $y \in f(A)$,
so $y \in f(A) \cup f(B)$.
In the second case, since $x \in B$, $y \in f(B)$,
so $y \in f(A) \cup f(B)$.

Since $y \in f(A \cup B)$ implies that $y \in f(A) \cup f(B)$,
we have $f(A \cup B) \subset f(A) \cup f(B)$.
\medskip
Next, we will show that $f(A) \cup f(B) \subset f(A \cup B)$.
In other words, we want to show that
if $y \in f(A) \cup f(B)$, then $y \in f(A \cup B)$.

Assume that $y \in f(A) \cup f(B)$.
By the definition of set union, we have two cases:
first, $y \in f(A)$; and second, $y \in f(B)$.
In the first case, $y = f(x)$ for some $x \in A$.
In the second case, $y = f(x)$ for some $x \in B$.
In either case, $x \in A \cup B$, so $y \in f(A \cup B)$.

Since $y \in f(A) \cup f(B)$ implies that $y \in f(A \cup B)$,
we have $f(A) \cup f(B) \subset f(A \cup B)$.
In conjunction with the previous part, we have
$f(A \cup B) = f(A) \cup f(B)$, as desired.
\bigskip
\item{} ii. false

Let $X = \{0,1\}$, $Y = \{0\}$, $A = \{0\} \subset X$, and
$B = \{1\} \subset X$.
Define $f(x) = 0$ for any $x \in X$.
Notice that $f(A \cap B) = f(\emptyset) = \emptyset$,
while $f(A) = f(B) = \{0\}$, so $f(A) \cap f(B) = \{0\}$,
which is nonempty.
\medskip
It is true that $f(A \cap B) \subset f(A) \cap f(B)$.
In other words, we can show that
if $y \in f(A \cap B)$, then $y \in f(A) \cap f(B)$.

Assume that $y \in f(A \cap B)$.
We want to show that $y \in f(A) \cap f(B)$.
Since $y \in f(A \cap B)$, $y = f(x)$ for some $x \in A \cap B$.
By the definition of set intersection, $x \in A$ and $x \in B$.
Since $x \in A$, $y \in f(A)$.
Since $x \in B$, $y \in f(B)$.
By the definition of set intersection, $y \in f(A) \cap f(B)$.
Therefore, $f(A \cap B) \subset f(A) \cap f(B)$, as desired.
\bye
