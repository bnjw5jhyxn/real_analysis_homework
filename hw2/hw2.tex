\def\real{{\bf R}}
\def\natural{{\bf N}}
\centerline{Steve Hsu\hfill homework 2}
\item{1.5.4.} a.

The function $f(x) = 2{x - a \over b - a} - 1$ maps $(a,b)$ to $(-1,1)$,
which we know has the same cardinality as $\real$.
\medskip
\item{} b.

The function $f(x) = 1 / (x + 1 - a)$ maps $(a,\infty)$ to $(0,1)$,
which by part a has the same cardinality as $\real$.
\medskip
\item{} c.

The function
$$f(x) = \cases{
1/2,&if $x = 0$\cr
x/2,&if $x = 2^{-n}$ for some $n \in \natural$\cr
x,&otherwise\cr
}$$
is a bijection between $[0,1)$ and $(0,1)$.
\bigskip
\item{1.5.7.} a.

$f(x) = (x,1/2)$
\medskip
\item{} b.

Suppose that $a = 0.a_1 a_2 a_3 \cdots$ is a nonterminating decimal expansion
of $a$ and $b = 0.b_1 b_2 b_3 \cdots$ and is a nonterminating decimal
expansion of $b$.
Then $f(a,b) = 0.a_1 b_1 a_2 b_2 \cdots$.

This function is not onto.
For example, no pair $(a,b)$ maps to $599/4950 = 0.121010101010 \cdots$.
If there were such a pair, it would be $a = 1/9 = 0.111 \cdots$ and
$b = 1/5 = 0.2000 \cdots$, but since we take nonterminating decimal
expansions, $f(a,b) = 0.11191919191919 = 277 / 2475$.
\bigskip
\item{1.5.8.}

Observe that $2$ is an upper bound for $B$
since any number $x$ in $B$ greater than $2$
would give the set $\{x\}$, which would have a sum greater than $2$.

Let $B_n = \{x \in B : 2^{1 - n} < x \le 2^{2 - n}\}$
for each natural number $n$.

Notice that each $B_n$ must have at most $2^n$ elements
since otherwise, we would have
$\sum _{x \in B_n} x > 2^n 2^{1 - n} = 2$,
which violates the definition of $B$.

Since each $B_n$ is finite, we can list all elements of $B$ by first listing
elements of $B_1$, then elements of $B_2$, and so on.
$B$ is therefore countable or finite.
\bigskip
\item{1.5.9.} a.

$\sqrt 2$ is a root of $x^2 - 2$

$\root 3 \of 2$ is a root of $x^3 - 2$

$\sqrt 2 + \sqrt 3$ is a root of $x^4 - 10x^2 + 1$
\medskip
\item{} b.

Since a polynomial of degree $n$ has at most $n + 1$ terms,
its coefficients can be written as an $(n + 1)$-tuple
$(a_n, a_{n - 1}, a_{n - 2}, \cdots, a_2, a_1, a_0)$,
where $a_n$ is a nonzero integer and the other components are integers.
Since the set of possible values of $a_n$ is countable
and the set of possible values of $a_{n - 1}$ is countable,
the set of possible pairs $(a_n, a_{n - 1})$ is also countable
by rearrangement of $\natural$ into an infinite matrix
as shown in exercise 1.5.3.

Inductively, since the set of $(k + 1)$-tuples
$(a_n, a_{n - 1}, \cdots, a_{n - k})$ is countable
and the set of possible values of $a_{n - k - 1}$ is countable,
the set of $(k + 2)$-tuples is also countable.

Finally, the set of $(n + 1)$ tuples of integers is countable,
so the set of polynomials of degree $n$ is also countable.
Since each polynomial has a finite number of roots,
the set $A_n$ of roots of these polynomials is countable as well.
\medskip
\item{} c.

By Theorem 1.5.8, the union of a sequence of countable sets is countable,
so $\bigcup _{n \in \natural} A_n$, which is the set of algebraic numbers,
is countable.

If the set of transcendental numbers were countable,
its union with the set of algebraic numbers would also be countable,
but we know that the set of real numbers is uncountable.
The set of transcendental numbers is therefore uncountable.
\bigskip
\item{1.6.4.}

Suppose for contradiction that $S$ is countable.
In other words, $S$ can be written as $\{s_1, s_2, s_3, \cdots\}$.
Write $s_n$ as $(s_{n1}, s_{n2}, s_{n3}, \cdots)$ for each natural number
$n$.
We define a sequence $a = (a_1, a_2, a_3, \cdots)$, where $a_n = 1$
if $s_{nn} = 0$ and $a_n = 0$ if $s_{nn} = 1$ for each natural number $n$.
Since $a$ differs from each $s_n$ at the $n$th entry,
$a$ is a sequence of $0$'s and $1$'s that is not in the sequence.
The sequence therefore does not include all members of $S$,
contradicting our assumption that $S$ is countable.
\bigskip
\item{1.6.9.}

We will define an injective function from $P(\natural)$ to $(0,1)$
and another injective function from $(0,1)$ to $P(\natural)$.
By the Schr\"oder-Bernstein Theorem, there will be a bijection
between the two sets.

Let $A$ be a set of natural numbers.
We define
$$f(A) = \cases{
{1 \over 2} \sum _{x \in A} 2^{-x},&if A is infinite\cr
\noalign{\vskip2pt}
{1 \over 2} (1 + \sum _{x \in A} 2^{-x}),&if A is finite\cr
}$$.

Let $a$ be a real number in the interval $(0,1)$ with nonterminating
binary expansion $0.a_1 a_2 a_3 \cdots$.
We define $g(a) = \{x \in \natural : a_x = 1\}$.

Since we have injective functions in both directions,
by the Schr\"oder-Bernstein Theorem, $P(\natural) \sim (0,1)$.
We know that $(0,1) \sim \real$, so $P(\natural) \sim \real$.
\bye
