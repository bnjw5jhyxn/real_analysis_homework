\def\abs#1{\vert#1\vert}
\def\real{{\bf R}}
\def\natural{{\bf N}}
\centerline{Steve Hsu\hfill homework 6}
\item{3.2.4.} a.

If $s \in A$, then since $A \subset \overline A$,
$s \in \overline A$, as desired.

Otherwise, since $s = \sup A$, for any $\epsilon > 0$,
$s - \epsilon$ is not an upper bound for $A$.
Therefore, there is some element $x \in A$ such that
$s - \epsilon < x < s$, so $\abs{x - s} < \epsilon$.
Therefore, $s$ is a limit point of $A$
and is consequently in $\overline A$, as desired.
\medskip
\item{} b. no

Since $s$ is an upper bound of $A$,
there is no $\epsilon > 0$ satisfying
$s + \epsilon/2 \in A$,
so there is no $\epsilon > 0$ satisfying
$(s - \epsilon, s + \epsilon) \subset A$.
Since $A$ is open, $s \notin A$.
\bigskip
\item{3.2.6.} a. false

The set $(-\infty,\sqrt 2) \cup (\sqrt 2, +\infty)$
is an open set containing every rational
that is not equal to $\real$.
\medskip
\item{} b. true

Let $F_1 \supset F_2 \supset \ldots$ be a sequence of
nonempty nested bounded closed sets.
Take a sequence $(x_n)$ where $x_n \in F_n$ for all natural numbers $n$.
Since $(x_n)$ is bounded, it has a convergent subsequence $(x_{n_k})$
by the Bolzano-Weierstrass Theorem.
Let $x = \lim x_{n_k}$.
Since $x_{n_\ell} \in F_{n_\ell} \subset F_{n_k} \subset F_n$
for all $\ell \ge k$,
eventually $x_{n_k} \in F_n$ for all natural numbers $n$.
Since $F_n$ is closed and $x$ is a limit point of $F_n$,
$x$ must also be in $F_n$.
Therefore, $x \in \bigcap _{k=1} ^\infty F_n$, so
$\bigcap _{k=1} ^\infty F_n \ne \emptyset$, as desired.
\medskip
\item{} c. true

Let $A$ be a nonempty open set and let $x \in A$.
Then there exists $\epsilon > 0$ such that
$(x - \epsilon, x + \epsilon) \subset A$.
Since the rational numbers are dense in the real numbers,
there is a rational number in this interval,
so $A$ contains a rational number, as desired.
\medskip
\item{} d. false

Let $F = \{\sqrt 2\} \cup \{(1 + 1/n)\sqrt 2: n \in \natural\}$.
Notice that $F$ is bounded by $3$
and that $\sqrt 2$ is the only limit point of $F$.
Clearly, $F$ is an bounded infinite closed set
that does not contain a rational number.
\medskip
\item{} e. true

Notice that a real number $a$ is in the Cantor set
if and only if it can be written in the form
$a = \sum _{k=1} ^\infty c_k ({1 \over 3})^k$,
where $c_k = 0$ or $c_k = 2$ for all natural numbers $k$.
Let $(x_n)$ be a convergent sequence of elements of the Cantor set.
Since $(x_n)$ is Cauchy by the Cauchy Criterion,
taking $\epsilon = ({1 \over 3})^K$ for any natural number $K$,
there exists a natural number $N$ such that the first $K$ terms of
$x_n = \sum _{k=1} ^\infty c_{nk} ({1 \over 3})^k$
are identical for all $n \ge N$.
In other words, for all natural numbers $k$, $c_{nk}$
is eventually constant with respect to $n$.
Let $c_k$ be this constant value, which must be either $0$ or $2$.
We will show that $x_n$ converges to $\sum _{k=1} ^\infty c_k ({1 \over 3})^k$.
Given $\epsilon > 0$, take $K$ such that $({1 \over 3})^K < \epsilon$.
Take $N$ such that for all $n \ge N$ and $k \le K$, $c_{nk} = c_k$.
Then we have that for $n \ge N$,
$\abs{(\sum _{k=1} ^\infty c_{nk} ({1 \over 3})^k) -
(\sum _{k=1} ^\infty c_k ({1 \over 3})^k)} =
\abs{\sum _{k=1} ^\infty (c_{nk} - c_k) ({1 \over 3})^k} =
\abs{\sum _{k=K+1} ^\infty (c_{nk} - c_k) ({1 \over 3})^k} \le
\sum _{k=K+1} ^\infty \abs{c_{nk} - c_k} ({1 \over 3})^k \le
\sum _{k=K+1} ^\infty 2 ({1 \over 3})^k = ({1 \over 3})^K < \epsilon$.
Therefore, $(x_n)$ converges to $\sum _{k=1} ^\infty c_k ({1 \over 3})^k$,
that is, $(x_n)$ converges to an element of the Cantor set.
Since any convergent sequence $(x_n)$ consisting of elements in the Cantor set
converges to an element of the Cantor set, the Cantor set is closed, as desired.
\bigskip
\item{3.2.12.}

Given a real number $s$, define $L_s = \{x \in A : x < s\}$
and $G_s = \{x \in A : x > s\}$.
Let $C_L$ be the set of real numbers $s$ such that $L_s$ is uncountable.
If $s \in C_L$, then for any nonnegative real number $t$,
$L_s \subset L_{s + t}$, so $s + t \in C_L$.
In other words, there is a number $y$ such that
$C_L = (y, +\infty)$ or $C_L = [y, +\infty)$.

We will show that $C_L = (y, +\infty)$.
Since $C_L = [y, +\infty)$ or $C_L = (y, +\infty)$,
we have that $y = \inf C_L$.
Then for any $\epsilon > 0$,
$L_{y - \epsilon} = \{x \in A : x < y - \epsilon\}$ is countable or finite.
Let $A_n = \{x \in A : x < y - 1/n\}$ for each natural number $n$.
Since $A_n$ is finite or countable for all $n$,
$\bigcup _{n=1} ^\infty A_n = \{x \in A : x < y\} = L_y$
is also finite or countable.
Therefore, $y \notin C_L$, so $C_L = (y, +\infty)$.

Similarly, let $C_G$ be the set of real numbers $s$
such that $G_s$ is uncountable.
By the same argument, there is a real number $z$
such that $C_G = (-\infty, z)$.

Since the union of two finite or countable sets is finite or countable,
and $L_s \cup G_s = A \setminus \{s\}$ is uncountable,
$L_s$ or $G_s$ is uncountable for any real number $s$.
Consequently, $C_L \cup C_G = \real$ and $y < z$.
Therefore, $B = C_L \cap C_R = (y,z)$, which is open, as desired.
\bigskip\goodbreak
\item{3.2.14.} a.

For the forward direction, assume that $E \subset \real$ is closed.
Then $E$ contains its set of limit points, $E'$.
Since $E' \subset E$, we have that $\overline E = E \cup E' = E$.

For the backward direction, assume that $\overline E = E$.
Let $E'$ be the set of limit points of $E$.
We have that $E' \subset E \cup E' = \overline E = E$,
so $E$ is closed, as desired.
\smallskip
For the forward direction, assume that $E$ is open.
We have that for all $x \in E$,
there exists $\epsilon > 0$ such that $V_\epsilon(x) \subset E$.
Therefore, the set of all $x \in E$ such that
there exists $\epsilon > 0$ such that $V_\epsilon(x) \subset E$
is simply all of $E$, that is, $E^\circ = E$.

For the backward direction, assume that $E^\circ = E$.
Then the set of $x \in E$ such that
there exists $\epsilon > 0$ such that $V_\epsilon(x) \subset E$
is all of $E$.
Therefore, we have that for all $x \in E$,
there exists $\epsilon > 0$ such that $V_\epsilon(x) \subset E$,
that is, $E$ is open, as desired.
\medskip
\item{} b.

By definition, $\overline E^c =
\{x \in \real : x \notin \overline E\}$, that is,
$\overline E^c$ is the set of real numbers $x$ such that
it is not the case that for all $\epsilon > 0$,
$V_\epsilon(x) \cap E \ne \emptyset$.
This is the set of real numbers $x$ such that there exists $\epsilon > 0$
such that $V_\epsilon(x) \cap E = \emptyset$.
Also by definition, $(E^c)^\circ$ is the set of real numbers $x$ such that
there exists $\epsilon > 0$ such that $V_\epsilon(x) \subset E^c$.
This is the set of real numbers $x$ such that there exists $\epsilon > 0$
such that $V_\epsilon(x) \cap E = \emptyset$,
which is equal to $\overline E^c$, as desired.
\smallskip
Let $A = E^c$.
We have from the part above that $\overline A^c = (A^c)^\circ$.
Therefore, $(E^circ)^c = ((A^c)^\circ)^c = (\overline A^c)^c =
\overline A = \overline{E^c}$, as desired.
\bye
