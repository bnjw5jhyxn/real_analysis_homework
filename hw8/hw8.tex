\def\rational{{\bf Q}}
\def\irrational{{\bf I}}
\def\real{{\bf R}}
\def\abs#1{\vert#1\vert}
\centerline{Steve Hsu\hfill homework 8}
\item{3.4.5.}

\proclaim Lemma 1. If $X \subset Y$, then $\overline X \subset \overline Y$.

We will first show that $X \prime \subset Y \prime$,
where $X \prime$ and $Y \prime$ are the sets of limit points
of $X$ and $Y$, respectively.
Let $a$ be a limit point of $X$.
Then there is a sequence $(a_n) \subset X \setminus \{a\}$
that converges to $a$.
Since $X \subset Y$, we have that $(a_n) \subset Y \setminus \{a\}$
and $(a_n)$ converges to $a$, so $a$ is a limit point of $Y$.
Since $X \subset Y$ and $X \prime \subset Y \prime$,
clearly $\overline X = X \cup X \prime \subset Y \cup Y \prime = \overline Y$,
as desired.

\proclaim Lemma 2. If $X \subset Y$ and the sets $Y$ and $Z$ are separated,
then $X$ and $Z$ are separated.

Since $Y$ and $Z$ are separated, we have that $\overline Y \cap Z = \emptyset$
and $Y \cap \overline Z = \emptyset$.
Since $X \subset Y$, we have by lemma 1 that
$\overline X \subset \overline Y$.
Then clearly, $\overline X \cap Z \subset \overline Y \cap Z = \emptyset$,
so $\overline X \cap Z = \emptyset$.
In addition, $X \cap \overline Z \subset Y \cap \overline Z = \emptyset$,
so $X \cap \overline Z = \emptyset$.
Therefore, $X$ and $Z$ are separated, as desired.

\proclaim Lemma 3. If $X$ and $Y$ are open and disjoint,
then they are separated.

The statement is equivalent to saying that for all open sets $X$ and $Y$,
if $X$ and $Y$ are disjoint, then they are separated.
We will prove this statement by contraposition, that is, we will show that
if $X$ and $Y$ are not separated, then they are not disjoint.
Assume without loss of generality that there is a point $a \in \overline X$
such that $a \in Y$.

We have two cases.
If $a \in X$, then $a \in X \cap Y$ and $X$ and $Y$ are not disjoint.
If $a$ is a limit point of $X$, take $\epsilon > 0$ such that
$V _\epsilon (a) \subset Y$.
Since $a$ is a limit point of $X$, there is a point $b \in X$
such that $b \in V _\epsilon (a)$ and $b \in X$.
Since $b \in X$ and $b \in V _\epsilon (a) \subset Y$,
$b \in X \cap Y$ and $X$ and $Y$ are not disjoint.
\smallskip
Since $U$ and $V$ are open disjoint sets, by lemma 3, they are separated.
Since $A \subset U$ and $U$ and $V$ are separated, by lemma 2,
we have that $A$ and $V$ are separated.
Since $B \subset V$ and $A$ and $V$ are separated, by lemma 2,
we have that $A$ and $B$ are separated, as desired.
\bigskip
\item{3.4.7.} a.

Let $x < y$ be rational numbers.
Take $r = x + {1 \over \sqrt 2} (y - x)$.
Notice that $r$ is irrational and that $x < r < y$.
The sets $A = (-\infty, r) \cap Q$ and $B = (r, +\infty) \cap Q$
are separated sets, with $x \in A$, $y \in B$, and $A \cup B = \rational$.
\medskip
\item{} b. yes

Let $x < y$ be irrational numbers.
Since the rational numbers are dense in the real numbers,
there is a rational number $r$ such that $x < r < y$.
The sets $A = (-\infty, r) \cap \irrational$ and
$B = (r, +\infty) \cap \irrational$ are separated sets,
with $x \in A$, $y \in B$, and $A \cup B = \real \cap \irrational$.
\bigskip
\item{3.4.8.} a.

Take $N$ large enough that $({1 \over 3})^n < \epsilon$.
Such an $N$ exists because $\lim ({1 \over 3^n}) = 0$.
Since the length of each interval in $C_N$ is smaller than $\epsilon$,
$x$ and $y$ cannot be in the same interval,
otherwise $y - x$ would be less than $\epsilon$.
\medskip
\item{} b.

Given $x, y \in C$ such that $x < y$.
Consider $N$ from part (a) such that $x$ and $y$
cannot belong to the same interval in $C_N$.
Since the intervals of $C_N$ are separated,
there is a real number $r$ such that $x < r < y$ and $r \notin C_N$.
Clearly, $r \notin C$,
so $A = (-\infty, r) \cap C$ and $B = (r, +\infty) \cap C$
are separated sets whose union is $C$ satisfying $x \in A$ and $y \in B$.
\bigskip
\item{3.4.9.} a.

For each natural number $n$, notice that $V _{\epsilon _n} (r_n)$ is open.
Since a union of open sets is open, we have that
$O = \bigcup _{n=1} ^\infty V _{\epsilon _n} (r_n)$ is open.
Therefore, $F = O^c$ is closed.

Since $O$ is a union of open intervals,
we can bound the length of $O$ with
$\abs O \le \sum _{n=1} ^\infty \abs{V _{\epsilon _n} (r_n)} =
\sum _{n=1} ^\infty ({1 \over 2})^n$, which converges absolutely to $1$.
Since $O$ has a finite length, $F = O^c$ must be nonempty.

Since every rational number $x$ is equal to $r_n$ for some natural number $n$,
we have that $x = r_n \in V _{\epsilon _n} (r_n) \subset O = F^c$,
so $x \notin F$.
Therefore, $F$ contains no rational numbers.
\medskip\goodbreak
\item{} b.

We will show by contradiction that $F$ contains no open intervals.
Suppose for contradiction that $F$ contains an open interval.
Let $x$ and $y$ be points in this interval such that $x < y$.
Since the rational numbers are dense in the real numbers,
we have that there is a rational number $r$ such that $x < r < y$.
By the definition of intervals, since $x$ and $y$ are in the interval,
$r$ is also in the interval.
Since $F$ contains no rational numbers, we have that $r \notin F$.
This is a contradiction since $r$ is in an interval contained in $F$.

\proclaim Lemma. If $X \subset Y$ and $Y$ is totally disconnected,
then $X$ is totally disconnected.

Let $a, b \in X$ be such that $a < b$.
Since $X \subset Y$, we have that $a, b \in Y$.
Since $Y$ is totally disconnected, there are sets $A$ and $B$
such that $a \in A$, $a \in B$, $A \cup B = Y$, and $A$ and $B$ are separated.
Take $C = A \cap X$ and $D = B \cap X$.
Clearly, $a \in A \cap X = C$ and $b \in B \cap X = D$.
Since $X \subset Y$, we have that $C \cup D = (A \cap X) \cup (B \cap X) =
(A \cup B) \cap X = Y \cap X = X$.
By lemma 2 from exercise (3.4.5), since $A$ and $B$ are separated
and $C \subset A$, $C$ and $B$ are separated.
By another application of lemma 2, since $C$ and $B$ are separated
and $D \subset B$, we have that $C$ and $D$ are separated, as desired.

Since $F \subset \irrational$ and
the irrational numbers are totally disconnected,
by the lemma, we have that $F$ is totally disconnected.
\medskip
\item{} c.

It is not possible to know that $F$ is perfect.
For example, take $(r_n)$ such that
if $n \equiv 0 \pmod 3$, then
$\sqrt 2 - ({1 \over 2})^n - ({1 \over 2})^{n+4} < r_n <
\sqrt 2 - ({1 \over 2})^n$;
if $n \equiv 1 \pmod 3$, then
$\sqrt 2 + ({1 \over 2})^n < r_n <
\sqrt 2 + ({1 \over 2})^n + ({1 \over 2})^{n+4}$,
and if $n \equiv 2 \pmod 3$, then $\abs{r_n - \sqrt 2} > 1/2^n$.
Notice that $\sqrt 2$ is both an isolated point of $F$ and a member of $F$.

We can modify the selection of $(\epsilon _n)$
to make $F$ perfect.
These $(\epsilon _n)$ will have several properties.
First, $\epsilon _n$ will be irrational for all natural numbers $n$.
Second, $\epsilon _n < 1 / 2^n$ for all $n$.
Third, $O_k^c = (\bigcup _{n=1} ^k V _{\epsilon _n} (r_n))^c$
can be written as a union of closed intervals with positive
or infinite length for all natural numbers $k$.

We give an inductive construction of $(\epsilon _n)$
that satisfies these properties.
Take $\epsilon _1 = \sqrt 2 / 3$.
Clearly, $\epsilon _1$ is irrational and is less than $1/2$.
Notice that $(V _{\epsilon _1} (r_1))^c =
(-\infty, r_1 - {\sqrt 2 \over 3}] \cup [r_1 + {\sqrt 2 \over 3}, +\infty)$.
With $\epsilon _1, \epsilon _2, \ldots, \epsilon _k$ fixed,
notice that $O_k = \bigcup _{n=1} ^k V _{\epsilon _n} (r_n)$
is a union of open intervals, so $O_k$ is open.
Since $\epsilon _n$ is irrational for $n \le k$,
we have that $r_n$ is not an endpoint of any of these intervals.

We therefore have two cases.
If $r_{k+1} \in O_k = \bigcup _{n=1} ^k V _{\epsilon n} (r_n)$,
then since $O_k$ is open, there exists $\delta _{k+1} > 0$
such that $V _{\delta _{k+1}} (r_{k+1}) \subset O_k$.
Take $\epsilon _{k+1} < \delta _{k+1}$
such that $\epsilon _{k+1}$ is irrational and less than $1 / 2^n$.
If $r_{k+1} \notin O_k$, since $O_k^c$ can be written as a union of closed
intervals with irrational endpoints, we have that $r_{k+1}$
is in the interior of $O_k^c$.
In this case, there exists $\delta _{k+1} > 0$
such that $V _{\delta _{k+1}} (r_{k+1}) \subset O_k^c$.
Take $\epsilon _{k+1} < {1 \over 2} \delta _{k+1}$
such that $\epsilon _{k+1}$ is irrational and less than $1 / 2^n$.

Notice that in the first case, $O_{k+1} = O_k$,
so we preserve the third property by the induction hypothesis.
In the second case, we simply split the interval of $O_k^c$
containing $r_{k+1}$.
Let this interval be $[a,b]$.
We know that $a \le r_{k+1} - \delta _{k+1} < r_{k+1} - \epsilon _{k+1} <
r_{k+1} < r_{k+1} + \epsilon _{k+1} < r_{k+1} + \delta _{k+1} \le b$.
Therefore, we split $[a,b]$ into $[a, r_{k+1} - \epsilon _{k+1}]$ and
$[r_{k+1} + \epsilon _{k+1}, b]$.
By the inequalities above, these intervals have positive length.
By a similar argument, we split an unbounded interval into another
unbounded interval and an interval with positive length.

We will now show that
if $x$ is an endpoint of an interval in $O_k^c$, then $x \in F$.
Let $x$ be an endpoint of an interval in $O_k^c$.
For any $r_n$ with $n > k$, we have two cases.

If $r_n$ is in the same interval as $x$,
since $r_n$ is rational and $x$ is irrational (by the definiton of $F$),
we have that $r_n \ne x$.
By our construction, we have that $\delta _n$ is small enough that
$V _{\delta _n} (r_n)$ is disjoint with $O_k$.
By lemma 3 from exercise (3.4.5),
since $V _{\delta _n} (r_n)$ and $O_k$ are open and disjoint,
they are separated.
Therefore, $x$, which is a limit point of $O_k$,
is not in $V _{\delta _n} (r_n)$.
Since $V _{\epsilon _n} (r_n) \subset V _{\delta _n} (r_n)$,
we have that $x \notin V _{\epsilon _n} (r_n)$.

If $r_n$ is not in the same interval as $x$,
then since its $\epsilon$-neighborhood cannot cross the boundaries
of intervals, $V _{\epsilon _n} (r_n)$ cannot include $x$.
Since $x \notin V _{\epsilon _n} (r_n)$ for all $n > k$
and $x \notin O_k$, we have that $x \notin O$, so $x \in F$, as desired.

By the same proof as in part (a),
$F$ constructed in this way is closed and nonempty.
We will now show that $F$ is perfect, that is,
we will show that every point $x \in F$
is a limit point of $F$.
For each natural number $n$,
take rational numbers $s$ and $t$ such that
$x - 1 < s < x$ and $x < t < x + 1$.
Since $\{r_1, \ldots\}$ is an enumeration of the rational numbers,
we have that $s = r_a$ and $t = r_b$ for some natural numbers $a$ and $b$.
Take $k = \max \{a, b\}$.
Since $x \in F = O^c = (\bigcup _{n=1} ^\infty V _{\epsilon _n} (r_n))^c =
\bigcap _{n=1} ^\infty (V _{\epsilon _n} (r_n))^c \subset
\bigcap _{n=1} ^k (V _{\epsilon _n} (r_n))^c =
(\bigcup _{n=1} ^k V _{\epsilon _n} (r_n))^c = O_k^c$,
by the properties established above,
$x$ is in a closed interval with positive length contained in $O_k^c$.
By the claim above both endpoints of this interval are in $F$
and since the interval has positive length, at least one of these endpoints
is not $x$.
Take $x_n$ to be this endpoint.
Clearly $(x_n) \subset F$ converges to $x$.
Therefore, $x$ is a limit point of $F$, as desired.
\bigskip
\item{4.2.1.} c.

By the sequential criterion for function limits, it suffices to show that
for all sequences $(x_n) \subset A$ that converge to $c$,
$(f(x_n) g(x_n))$ converges to $LM$.
Let $(x_n) \subset A$ be a sequence that converges to $c$.
Since $\lim _{x \to c} f(x) = L$,
by the sequential criterion for functional limits,
we have that $(f(x_n))$ converges to $L$.
By the same argument, $(g(x_n))$ converges to $M$.
By the algebraic limit theorem for sequences, we have that
$(f(x_n) g(x_n))$ converges to $LM$, as desired.
\medskip
Given $\epsilon > 0$, take $\delta _1$ such that
for all $x$ such that $\abs{x - c} < \delta _1$,
$\abs{f(x) - L} < \epsilon / (2 \abs M)$,
take $\delta _2$ such that
$\abs{f(x) - L} < 1$, and
take $\delta _3$ such that
$\abs{g(x) - M} < \epsilon / (2 \abs L + 2)$.
Let $\delta = \min \{\delta _1, \delta _2, \delta _3\}$.
Notice that if $\abs{x - c} < \delta$, then
$\abs{f(x) g(x) - LM} = \abs{f(x) g(x) - f(x) M + f(x) M - LM} \le
\abs{f(x)} \abs{g(x) - M} + \abs M \abs{f(x) - L} <
(\abs L + 1) \epsilon / (2 \abs L + 2) + \abs M \epsilon / (2 \abs M) =
\epsilon / 2 + \epsilon / 2 = \epsilon$, as desired.
\bigskip
\item{4.2.5.} a.

Given $\epsilon > 0$, take $\delta = \epsilon / 3$.
If $\abs{x - 2} < \epsilon / 3$, then
$\abs{3x + 4 - 10} = \abs{3x - 6} = 3 \abs{x - 2} <
3 (\epsilon / 3) = \epsilon$, as desired.
\medskip
\item{} b.

Given $\epsilon < 0$, take $\delta = \min \{1, \epsilon\}$.
If $\abs{x - 0} = \abs x < \delta$, then
$\abs{x^3 - 0} = \abs{x^3} = \abs x \abs x \abs x < (1) (1) \abs x = \abs x <
\delta \le \epsilon$, as desired.
\medskip
\item{} c.

Given $\epsilon > 0$, take $\delta = \min \{1, \epsilon / 6\}$.
Notice that if $\abs{x - 2} < 1$, then $x < 3$ and $x + 3 < 6$.
If $\abs{x - 2} < \delta$, then
$\abs{x^2 + x - 1 - 5} = \abs{x^2 + x - 6} = \abs{(x + 3)(x - 2)} =
\abs{x + 3} \abs{x - 2} < 6 (\epsilon / 6) = \epsilon$, as desired.
\medskip
\item{} d.

Given $\epsilon > 0$, take $\delta = \min \{1, 2 \epsilon\}$.
Notice that if $\abs{x - 3} < 1$, then $x > 2$ and $1/x < 1/2$.
If $\abs{x - 3} < \delta$, then
$\abs{{1 \over x} - {1 \over 3}} = \abs{(3 - x) / (3x)} =
\abs{1/3x} \abs{3 - x} < (1/2) (2 \epsilon) = \epsilon$, as desired.
\bye
