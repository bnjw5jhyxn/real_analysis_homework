\def\abs#1{\vert#1\vert}
\def\real{{\bf R}}
\def\natural{{\bf N}}
\centerline{Steve Hsu\hfill homework 7}
\item{3.3.4.} a. both compact and closed

Since $K$ is compact, $K$ is closed and bounded.
We have that $K \cap F$ is also closed since
it is an intersection of closed sets.
Since $K$ is bounded, there is a positive real number $M$
such that $\abs x \le M$ for all $x \in K$.
Notice that $\abs x \le M$ for all $x \in K \cap F$,
so $K \cap F$ is bounded.
Since $K \cap F$ is closed and bounded, it is compact.
\medskip
\item{} b. closed

Since $\overline{F^c \cup K^c}$ is the closure of a set, it is closed.
It is not necessarily bounded.
For example, take $F = K = \emptyset$.
Then $\overline{F^c \cup K^c} = \overline{\emptyset^c \cup \emptyset^c} =
\overline{\real \cup \real} = \overline{\real} = \real$,
which is clearly not bounded.
\medskip
\item{} c. neither.

Take $K = [0,1]$ and $F = [1,2]$.
Then $K \setminus F = [0,1)$, which is not closed and not compact.
\medskip
\item{} d. both compact and closed

\proclaim Lemma. If a set $A$ is bounded, then $\overline A$ is also bounded.

Let $a$ be an upper bound for $A$.
We will show that $a$ is also an upper bound for $A'$,
the set of limit points of $A$,
that is, we will show that $\lim x_n \le a$
for all convergent sequences $(x_n) \subset A$.
Let $(x_n) \subset A$ be a convergent sequence.
Since $a$ is an upper bound for $a$,
we have that $x_n \le a$ for all $n \in \natural$.
By the order limit theorem, $\lim x_n \le a$.
Since $a$ is an upper bound for both $A$ and $A'$,
it follows that $a$ is an upper bound for $\overline A = A \cup A'$,
so $\overline A$ is bounded above.
By the same argument, $\overline A$ is bounded below,
so $\overline A$ is bounded, as desired.
\smallskip
Since $\overline{K \cap F^c}$ is the closure of a set, it is closed.
By the argument in part (a), since $K$ is bounded,
$K \cap F^c$ is also bounded.
By the lemma, it follows that $\overline{K \cap F^c}$ is also bounded.
Since $\overline{K \cap F^c}$ is closed and bounded,
it is compact, as desired.
\bigskip
\item{3.3.6.} a.

The empty set, which is finite, compact, and closed, does not have a maximum.

Every nonempty finite set has a maximum.
We will show by induction that for all natural numbers $n$,
every finite set with $n$ elements has a maximum.
For the base case, let $A$ be a set with exactly one element.
Let $x$ be this element.
Then $x = \max A$.
For the inductive step, assume that
every set with exactly $n$ elements has a maximum.
Let $A$ be a set with exactly $n + 1$ elements.
Then we can write $A = \{x_1, x_2, \ldots, x_n, x_{n+1}\}$.
Let $B = \{x_1, \ldots, x_n\}$.
Notice that $B$ has exactly $n$ elements and that $A = B \cup \{x_{n+1}\}$.
By the induction hypothesis, $B$ has a maximum.
Let $y = \max B$.
We have two cases.
If $y \ge x_{n+1}$, then $y$ is an upper bound for $A$.
Since $y = \max B$, we have that $y \in B \subset A$, so $y \in A$.
Therefore, $y = \max A$.
If $x_{n+1} \ge y$, then $x_{n+1} \ge y \ge b$ for all $b \in B$,
and $x_{n+1} \ge x_{n+1}$,
so $x_{n+1}$ is an upper bound for $A$.
Since $x_{n+1}$ is an upper bound for $A$ and $x_{n+1} \in A$,
we have that $x_{n+1} = \max A$.
Since $\max A$ exists in either case, $A$ has a maximum.

Every nonempty compact set $K$ has a maximum.
Since $K$ is compact, $K$ is closed and bounded.
Let $a = \sup K$.
We will show by contradiction that $a \in K$.
Assume for contradiction that $a \notin K$.
Then given $\epsilon > 0$, since $a = \sup A$,
we have that $a - \epsilon$ is not an upper bound for $A$.
Therefore, there exists $x \in A$ such that $a - \epsilon < x < a$,
that is, $x \ne a$ and $\abs{x - a} < \epsilon$.
Therefore, $a$ is a limit point of $K$.
Since $K$ is closed and $a$ is a limit point of $A$,
$A \in K$, contradicting our assumption that $a \notin K$.
Since $a$ is an upper bound for $K$ and $a \in K$,
we have that $a = \max K$, as desired.

A nonempty closed set does not necessarily have a maximum.
The interval $[0,\infty)$ is closed and has no maximum.
\medskip
\item{} b.

\proclaim Lemma. Let $S$ and $T$ be sets.
If $S$ is finite and there is an onto function
$f: S \to T$, then $T$ is finite.

Since $S$ is finite, we can write
$S = \{x_1, x_2, \ldots, x_n\}$, where $n$ is the number of elements of $S$.
Since $f$ is onto, we have that $T = f(S) =
\{f(x_1), f(x_2), \ldots, f(x_n)\}$, which is clearly finite.
\smallskip
Notice that there is an onto function $f: A \times B \to A+B$,
defined by $f(x,y) = x + y$.
Since $A$ and $B$ are finite,
let $m$ be the number of elements of $A$ and
let $n$ be the number of elements of $B$.
Then $A \times B$ has $mn$ elements, so it is finite.
By the lemma, we have that $A + B = f(A \times b)$
is also finite, as desired.

We will show that if $A$ and $B$ are compact,
then $A + B$ is also compact, that is,
every sequence $(x_n) \subset A + B$ has a subsequence
converging to a point in $A + B$.
Let $(x_n) \subset A + B$ be a sequence.
By definition of $A + B$, we can write $x_n = a_n + b_n$,
where $a_n \in A$ and $b_n \in B$ for each natural number $n$.
Since $A$ is compact, $(a_n)$ has a subsequence $(a_{n_k})$
that converges to a point $a \in A$.
Since $B$ is compact, $(b_{n_k})$ has a subsequence $(b_{n_{k_\ell}})$
that converges to a point $b \in B$.
Notice that $(a_{n_{k_\ell}})$ also converges to $a$.
Therefore, $(x_{n_{k_\ell}}) = (a_{n_{k_\ell}} + b_{n_{k_\ell}})$
converges to $a + b$, which is in $A + B$.

If $A$ and $B$ are closed, $A + B$ is not necessarily closed.
Take $A = \{n + 1/n : n \in \natural, n \ge 2\}$ and
$B = \{-n : n \in \natural, n \ge 2\}$.
Since $A$ and $B$ have no limit points, they are closed.
Notice that the sequence $(n + 1/n - n) = (1/n) \subset A+B$,
so $0$ is a limit point of $A + B$.
Since all elements of $A$ are non-integers and
all elements of $B$ are integers,
there are no elements $a \in A$ and $b \in B$ such that $a + b = 0$,
so $A + B$ does not contain $0$, which is a limit point of $A + B$.
\medskip
\item{} c.

\proclaim Lemma. Let $A_1 \supset A_2 \supset \cdots$
be a nested sequence of finite sets.
There is a natural number $k$ such that
$A_n = A_k$ for all $n$ such that $n \ge k$.

Since $A_1$ is finite,
there are finitely many natural numbers between $|A_1|$ and $0$.
Since $A_n \subseteq A_{n-1}$ for each natural number $n \ge 2$,
$|A_n| \le |A_{n-1}|$.
Therefore, there are finitely many natural numbers $n$ such that
$|A_n| < |A_{n-1}|$.
Let $k$ be the largest such $n$.
We can now conclude that for each $n$ such that $n \ge k$,
$|A_n| = |A_k|$, and since $A_n \subseteq A_k$,
we have $A_n = A_k$, as desired.
\smallskip
Let $\{A_n : n \in \natural\}$ be a collection of finite sets.
Let $B_n = \bigcap _{k=1} ^n A_k$.
Notice that $B_n = \bigcap _{k=1} ^n A_k$ is nonempty by the problem statement
and that $B_1 \supset B_2 \supset \cdots$ is a nested sequence of finite sets.
By the lemma, there is a natural number $k$ such that
$B_n = B_k$ for all $n \ge k$.
Then $\bigcap _{n=1} ^\infty A_n = \bigcap _{n=1} ^\infty B_n =
\bigcap _{n=1} ^k B_n = B_k$ is nonempty, as desired.

Let $\{A_n : n \in \natural\}$ be a collection of compact sets.
Let $B_n = \bigcap _{k=1} ^n A_k$.
Notice that $B_n = \bigcap _{k=1} ^n A_k$ is nonempty by the problem statement,
that $B_n$ is compact since it is the intersection of compact sets,
and that $B_1 \supset B_2 \supset \cdots$ is a nested sequence of compact sets.
By the nested compact set property, we have that
$\bigcap _{n=1} ^\infty A_n = \bigcap _{n=1} ^\infty B_n$
is nonempty, as desired.

Let $\{A_n : n \in \natural\}$ be a collection of closed sets.
Then $\bigcap _{n=1} ^\infty A_n$ is not necessarily nonempty.
Take $A_n = [n, \infty)$.
By the Archimedean property, for any real number $x$,
there is a natural number $n$ such that $n > x$.
Therefore, $x \notin A_n$, so $x \notin \bigcap _{n=1} ^\infty A_n$,
so $\bigcap _{n=1} ^\infty A_n$ is empty, as desired.
\bye
