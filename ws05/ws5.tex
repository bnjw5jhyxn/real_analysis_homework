\def\natural{{\bf N}}
\centerline{Steve Hsu\hfill workshop 5}
\item{2.} a.
\itemitem{$n = 1$)} $2^0 = 1 \le 1 < 2 = 2^1$, so $k = 0$
\itemitem{} $a_1 = 1 + 1 - 2^0 = 2 - 1 = 1$
\itemitem{$n = 2$)} $2^1 = 2 \le 2 < 4 = 2^2$, so $k = 1$
\itemitem{} $a_2 = 2 + 1 - 2^1 = 3 - 2 = 1$
\itemitem{$n = 3$)} $2^1 = 2 \le 3 < 4 = 2^2$, so $k = 1$
\itemitem{} $a_3 = 3 + 1 - 2^1 = 4 - 2 = 2$
\itemitem{$n = 4$)} $2^2 = 4 \le 4 < 8 = 2^3$, so $k = 2$
\itemitem{} $a_4 = 4 + 1 - 2^2 = 5 - 4 = 1$
\itemitem{$n = 5$)} $2^2 = 4 \le 5 < 8 = 2^3$, so $k = 2$
\itemitem{} $a_5 = 5 + 1 - 2^2 = 6 - 4 = 2$
\itemitem{$n = 4$)} $2^2 = 4 \le 6 < 8 = 2^3$, so $k = 2$
\itemitem{} $a_6 = 6 + 1 - 2^2 = 7 - 4 = 3$
\itemitem{$n = 7$)} $2^2 = 4 \le 7 < 8 = 2^3$, so $k = 2$
\itemitem{} $a_7 = 7 + 1 - 2^2 = 8 - 4 = 4$
\itemitem{$n = 8$)} $2^3 = 8 \le 8 < 16 = 2^4$, so $k = 3$
\itemitem{} $a_8 = 8 + 1 - 2^3 = 9 - 8 = 1$
\itemitem{$n = 9$)} $2^3 = 8 \le 9 < 16 = 2^4$, so $k = 3$
\itemitem{} $a_9 = 9 + 1 - 2^3 = 10 - 8 = 2$
\medskip
\item{} b.

First, we will show that $\natural \subset E$,
that is, for every natural number $x$,
there is a subsequence $(a_{n_k})$ of $(a_n)$
that converges to $x$.
Given a natural number $x$,
find the unique integer $y$ such that $2^y \le x < 2^{y + 1}$.
Let $n_k = x - 1 + 2^k$.
For $k > y$, $k$ satisfies $2^k \le n_k < 2^{k + 1}$
since $x - 1 < 2^{y + 1}$.
Therefore, $a_{n_k} = n_k + 1 - 2^k = x - 1 + 2^k + 1 - 2^k = x$,
so $(a_{n_k})$ is eventually constant and therefore converges to $x$, as desired.

Next, we will show that $E \subset \natural$,
that is, for every convergent subsequence $(a_{n_k})$ of $(a_n)$,
$\lim _{k \to \infty} a_{n_k}$ is a natural number.
Given a convergent subsequence $(a_{n_k})$, let $a = \lim _{k \to \infty} a_{n_k}$.
Take $\epsilon = 1/2$.
Since $(a_{n_k}) \to a$, $(a_{n_k})$ is eventually in the interval
$(a - 1/2, a + 1/2)$.
Notice that there is at most one integer in $(a - 1/2, a + 1/2)$.
Since $a_{n_k}$ is an integer for all $k$,
there is such an integer.
Call this integer $x$.
Since $(a_{n_k})$ is eventually in the interval $(a - 1/2, a + 1/2)$,
and $a_{n_k}$ is an integer for all $k$,
$(a_{n_k})$ is eventually constant and must therefore converge to $x$,
which is a natural number, as desired.
\bye
