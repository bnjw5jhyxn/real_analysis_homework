\def\abs#1{\vert{#1}\vert}
\overfullrule=0pt
\centerline{Steve Hsu\hfill homework 4}
\item{2.3.13.} a.

We will show that for fixed $n$, $\lim _{m \to \infty} a_{mn} = 1$.
Given $\epsilon > 0$, take $M > n(1/\epsilon - 1)$.
Note that we can manipulate this inequality to get $n/(M + n) < \epsilon$.
Then for $m \ge M$, we have $\abs{1 - m/(m + n)} =
1 - m/(m + n) = n/(m + n) \le n/(M + n) < \epsilon$, as desired.
Then $\lim _{n \to \infty} (\lim _{m \to \infty} a_{mn}) = \lim _{n \to \infty} (1) = 1$.

For fixed $m$, $\lim _{n \to \infty} a_{mn} = 0$.
Given $\epsilon > 0$, take $N > m(1/\epsilon - 1)$.
Note that we can manipulate this inequality to get $m/(m + N) < \epsilon$.
Then for $n \ge N$, we have $\abs{m/(m + n) - 0} = m/(m + n) \le
m/(m + N) < \epsilon$, as desired.
Then $\lim _{m \to \infty} (\lim _{n \to \infty} a_{mn}) = \lim _{m \to \infty} (0) = 0$.
\medskip
\item{} b.

For $a_{mn} = 1/(m + n)$, $\lim _{m,n \to \infty} a_{mn} = 0$.
Given $\epsilon > 0$, take $N > 1/(2\epsilon)$.
Then for any $m,n \ge N$, $m + n > 1/\epsilon$ and
$\abs{1/(m + n) - 0} = 1/(m + n) < \epsilon$, as desired.

For fixed $n$, $\lim _{m \to \infty} a_{mn} = 0$.
Given $\epsilon > 0$, take $M > 1/\epsilon - n$.
Then for any $m \ge M$, we have $m + n \ge M + n > 1/\epsilon$,
so $\abs{1/(m + n) - 0} = 1/(m + n) < \epsilon$, as desired.
Therefore, $\lim _{n \to \infty} (\lim _{m \to \infty} a_{mn}) = 0$.
By the same argument, $\lim _{m \to \infty} (\lim _{n \to \infty} a_{mn}) = 0$.
\smallskip
For $a_{mn} = mn/(m^2 + n^2)$, $\lim _{m,n \to \infty}$ does not exist.
Take $\epsilon = 1/20$.
For any natural number $N$, $a_{NN} = N^2 / (N^2 + N^2) = 1/2$,
while $a_{N3N} = 3N^2 / (N^2 + 9N^2) = 3/10$.
Consequently, $a_{mn}$ is not Cauchy and does not converge.

For fixed $m$, $\lim _{n \to \infty} a_{mn} = 0$.
Given $\epsilon > 0$, take $N > m/\epsilon$.
For any $n \ge N$, $n^2 > mn/\epsilon$, so $n^2 + m^2 > mn/\epsilon$,
and $\abs{mn/(m^2 + n^2) - 0} = mn/(m^2 + n^2) < \epsilon$, as desired.
Therefore, $\lim _{m \to \infty} (\lim _{n \to \infty} a_{mn}) = 0$.
By the same argument, $\lim _{m \to \infty} (\lim _{n \to \infty} a_{mn}) = 0$.
\medskip
\item{} c.

Let $a_{mn} = {\sin m \over n} + {\sin n \over m}$.

$\lim _{m,n \to \infty} a_{mn} = 0$.
Given $\epsilon > 0$, take $N > 2/\epsilon$.
For any $m,n \ge N$, note that $\abs {\sin m} \le 1$ and $\abs {\sin n} \le 1$.
Therefore, $\abs{{\sin m \over n} + {\sin n \over m}} \le
\abs{\sin m \over n} + \abs{\sin n \over m} \le 1/m + 1/n <
\epsilon/2 + \epsilon/2 = \epsilon$.

For fixed $n$, $\lim _{m \to \infty} a_{mn}$ does not exist.
Take $\epsilon = 1 / (2n)$.
For any natural number $M$,
there are values of $m \ge M$ such that $\sin m > 2/3$ and
values of $m \ge M$ such that $\sin m < -2/3$.
Since $({\sin m \over n}) \to 0$,
$a_{mn} < -1 / (2n)$ for some $m \ge M$ and
$a_{mn} > 1 / (2n)$ for some $m \ge M$.
Hence, $a_{mn}$ is not Cauchy and therefore diverges.
\medskip
\item{} d.

We want to show that $(b_m) \to a$.
Given $\epsilon > 0$, take $N$ such that
for $m,n \ge N$, $\abs{a_{mn} - a} < \epsilon / 2$.
For $m \ge N$, since $a_{mn}$ is eventually
in the interval $(a - \epsilon / 2, a + \epsilon / 2)$,
and $b_m = \lim _{n \to \infty} a_{mn}$ exists,
by the order limit theorem, $a - \epsilon / 2 \le b_m \le a + \epsilon / 2$.
Therefore, $\abs {b_m - a} \le \epsilon$, as desired.
\medskip
\item{} e.

Let $a = \lim _{m,n \to \infty} a_{mn}$.
Consider the sequence $(b_m)$
where $b_m = \lim _{n \to \infty} a_{mn}$.
\hfil\break
By part (d), $\lim _{m \to \infty} (\lim _{n \to \infty} a_{mn}) =
\lim _{m \to \infty} b_m = a$.
Similarly, consider the sequence $(c_n)$
where $c_n = \lim _{m \to \infty} a_{mn}$.
By part (d), $\lim _{n \to \infty} (\lim _{m \to \infty} a_{mn}) =
\lim _{n \to \infty} c_n = a$.
\bigskip
\item{2.4.3.} a.

Let $a_1 = \sqrt 2$ and $a_{n + 1} = \sqrt {2 + a_n}$.
We will show by induction that $a_n < 2$ for all natural numbers $n$.
$a_1 = \sqrt 2$ is clearly less than $2$.
Assuming $a_n < 2$, $a_{n + 1} = \sqrt {2 + a_n} < \sqrt {2 + 2} = 2$.
In addition, $(a_n)$ is strictly increasing.
We know that $a_n < 2$ for all $n$, so
$a_{n + 1} = \sqrt {2 + a_n} > \sqrt {a_n + a_n} = \sqrt {2a_n} > \sqrt {a_n ^2} = a_n$.
Since $(a_n)$ is a strictly increasing sequence that is bounded above,
by the monotone convergence theorem, $a = \lim a_n$ exists.
By taking the limit of both sides of the identity
$a_{n + 1} = \sqrt {2 + a_n}$, we obtain $a = \sqrt {2 + a}$.
Solving the equation, $a = 2$ or $a = -1$.
Since $(a_n)$ is a strictly increasing sequence whose first term is positive,
$a_n$ must be positive for all $n$.
By the order limit theorem, $a$ must also be positive, so $a = 2$.

\item{} b.

Let $a_1 = \sqrt 2$ and $a_{n + 1} = \sqrt {2a_n}$.
We will show by induction that $a_n < 2$ for all natural numbers $n$.
$a_1 = \sqrt 2$ is clearly less that $2$.
Assuming $a_n < 2$, $a_{n + 1} = \sqrt {2a_n} < \sqrt {(2)(2)} = 2$.
In addition, $(a_n)$ is strictly increasing.
We know that $a_n < 2$ for all $n$, so
$a_{n + 1} = \sqrt {2a_n} > \sqrt {a_n a_n} = a_n$.
Since $(a_n)$ is a strictly increasing sequence that is bounded above,
by the monotone convergence theorem, $a = \lim a_n$ exists.
By taking the limit of both sides of the identity
$a_{n + 1} = \sqrt {2a_n}$, we obtain $a = \sqrt {2a}$.
Solving the equation, $a = 2$ or $a = 0$.
Since $(a_n)$ is a strictly increasing sequence whose first term is $\sqrt 2$,
$a_n \ge \sqrt 2$ for all $n$.
By the order limit theorem, $a \ge \sqrt 2$ as well, so $a = 2$.
\bigskip
\item{2.4.6.} a.

Since $x$ and $y$ are real, ${1 \over 4} (x - y)^2 \ge 0$.
Expanding the left side, ${1 \over 4} x^2 - {1 \over 2} xy + {1 \over 4} y^2 \ge 0$.
Adding $xy$ to both sides, ${1 \over 4} x^2 + {1 \over 2} xy + {1 \over 4} y^2 \ge xy$.
Since $x$ and $y$ are positive, we can take the square root of both sides to obtain
$(x + y) / 2 \ge \sqrt {xy}$.
\medskip
\item{} b.

We will show that $x_n \le y_n$ for all natural numbers $n$.
It is given that $x_1 \le y_1$.
For $n \ge 1$, since $x_{n + 1}$ is the geometric mean of $x_n$ and $y_n$,
while $y_{n + 1}$ is the arithmetic mean of the same two numbers,
by part (a), $x_{n + 1} \le y_{n + 1}$.
We will now show that $(x_n)$ is increasing.
Since $x_n \le y_n$, $x_{n + 1} = \sqrt {x_n y_n} \ge \sqrt {x_n x_n} = x_n$.
We will also show that $(y_n)$ is decreasing.
Since $x_n \le y_n$, $y_{n + 1} = {1 \over 2} (x_n + y_n) \le {1 \over 2} (y_n + y_n) = y_n$.
In addition, $x_n \le y_1$ for all natural numbers $n$
since $x_n \le y_n \le y_1$, so $(x_n)$ is bounded above.
Similarly, $y_n \ge x_n \ge x_1$ for all natural numbers $n$,
so $(y_n)$ is bounded below.
Since $(x_n)$ is an increasing sequence that is bounded above,
$x = \lim x_n$ exists and since $(y_n)$ is a decreasing sequence
that is bounded below, $y = \lim y_n$ exists as well.
Taking the limit of both sides of the identity $y_{n + 1} = {1 \over 2} (x_n + y_n)$,
we obtain $y = {1 \over 2} (x + y)$.
Simplifying, we obtain $x = y$, as desired.
\bigskip
\item{2.4.7.} a.

Let $X_n = \{a_k : k \ge n\}$.
Since $X_{n + 1} \subset X_n$, any upper bound for $X_n$
is also an upper bound for $X_{n + 1}$.
In particular, $y_n = \sup X_n$ is an upper bound for $X_{n + 1}$
and consequently, $\sup X_n \ge \sup X_{n + 1}$.
Therefore $(y_n)$ is decreasing.
Since $(a_n)$ is bounded, there is a positive real number $M$
such that $a_n \ge -M$ for all natural numbers $n$.
Therefore, since $X_n$ is a nonempty set whose members
are all at least $-M$, $\sup X_n \ge -M$ for all natural numbers $M$.
Since $(y_n)$ is decreasing and bounded below,
it converges by the monotone convergence theorem.
\medskip
\item{} b.

Let $z_n = \inf X_n$ where $X_n$ is defined as in part (a).
The sequence $(z_n)$ is increasing and bounded above
by the same argument as in part (a),
so $\liminf a_n = \lim z_n$ exists.
\medskip
\item{} c.

Since $a_n \in X_n$ as defined in part (a),
$z_n = \inf X_n \le a_n \le \sup X_n = y_n$.
Since $z_n \le y_n$ for all natural numbers $n$,
by the order limit theorem, $\liminf a_n = \lim z_n \le \lim y_n = \limsup a_n$,
as desired.

Let $(a_n) = (1, -1, 1, -1, \cdots)$.
Clearly, $\liminf a_n = -1$ and $\limsup a_n = 1$,
so in this case, $\liminf a_n < \limsup a_n$.
\medskip
\item{} d.

For the forward direction, we want to show that
if $\liminf a_n = \limsup a_n = a$, then $\lim a_n = a$.
Let $y_n = \sup X_n$ and $z_n = \inf X_n$ as defined in part (a).
Then, as we observed in part (c), $z_n \le a_n \le y_n$.
Since $\lim z_n = \lim y_n = a$, by the squeeze theorem,
$\lim a_n = a$, as desired.

For the backward direction, we want to show that
if $\lim a_n = a$, then $\liminf a_n = \limsup a_n = a$.
Given $\epsilon > 0$, take $N$ such that
$\abs {a_n - a} \le \epsilon / 2$ for all $n \ge N$.
Then $X_n \subset (a - \epsilon / 2, a + \epsilon / 2)$
for $n \ge N$ and $X_n$ as defined in part (a).
Then $\sup X_n$ and $\inf X_n$ are both in the interval
$[a - \epsilon / 2, a + \epsilon / 2]$, so
$y_n$ and $z_n$ as defined in parts (a) and (b)
are both in $[a - \epsilon / 2, a + \epsilon / 2]$.
Therefore, $\abs {y_n - a} < \epsilon$
and $\abs{z_n - a} < \epsilon$,
so $\limsup a_n = \lim y_n = a$ and
$\liminf a_n = \lim z_n = a$, as desired.
\bigskip
\item{4.A.}

We will show that $\limsup a_n \in A$, that is,
there is a subsequence $(a_{n_k})$ of $(a_n)$
that converges to $\limsup a_n$.
Let $n_1 = 1$, that is, $a_{n_1} = a_1$.
Given $n_k$, let $n_{k + 1}$ be such that
$n_{k + 1} > n_k$ and $\abs{\sup X_{1 + n_k} - a_{n_{k + 1}}} < 1/(k + 1)$,
where $X_n$ is defined as in exercise 2.4.7.
Such an $n_{k + 1}$ exists because $\sup X_{1 + n_k} - 1/(k + 1)$
is not an upper bound for $X_{1 + n_k}$.
Notice that $X_{n_{k + 1}} \subset X_{1 + n_k}$,
so $\sup X_{1 + n_k} \ge \sup X_{n_{k + 1}} \ge a_{n_{k + 1}}$.
Therefore, $\abs{\sup X_{n_{k + 1}} - a_{n_{k + 1}}} < 1/(k + 1)$.
We claim that $\lim _{k \to \infty} a_{n_k} = \limsup a_n$.
Given $\epsilon > 0$, take $K$ such that $K > 2 / \epsilon$
and $\abs {\sup X_{n_k} - \limsup a_n} < \epsilon/2$ for any $k \ge K$.
Such a $K$ exists because $(\sup X_{n_k}) \to \limsup a_n$ by definition.
Note that $\abs {a_{n_K} - \sup X_{n_K}} < 1/K < \epsilon/2$.
For any $k \ge K$, $\abs{a_{n_k} - \limsup a_n} \le
\abs{a_{n_k} - \sup X_{n_k}} + \abs{\sup X_{n_k} - \limsup a_n} <
\epsilon/2 + \epsilon/2 = \epsilon$, as desired.

We will now show that $\limsup a_n$ is an upper bound for $A$.
Let $(a_{n_k})$ be any convergent subsequence of $(a_n)$.
Since $n_k \ge k$ for all natural numbers $k$,
$a_{n_k} \in X_k$ and $y_k = \sup X_k \ge a_{n_k}$.
Therefore, by the order limit theorem,
$\limsup a_n = \lim y_k \ge \lim a_{n_k}$, as desired.

Since $\limsup a_n$ is an element of $A$ and also and upper bound for $A$,
$\limsup a_n$ is the maximum of $A$ and is therefore equal to $\sup A$.
\bigskip
\item{2.5.5.}

Let $A$ be the set of limits of subsequences $(a_{n_k})$ of $(a_n)$
as defined in exercise 4.A.
Notice that $A = \{a\}$.
By exercise 4.A, $\limsup a_n = \sup A = a$ and,
by a similar argument, $\liminf a_n = \inf A = a$.
By exercise 2.4.7, $\lim a_n$ exists and is equal to $a$.
\bye
