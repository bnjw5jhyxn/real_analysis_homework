\def\abs#1{\vert{#1}\vert}
\centerline{Steve Hsu\hfill homework 5}
\item{2.4.10.} a.

If $a_n = 1/n$, then
$p_m = \prod _{n=1} ^m (1 + 1/n) = \prod _{n=1} ^m (n+1)/n =
(2/1)(3/2)\cdots((m+1)/m) = m+1$, which clearly diverges.

If $a_n = 1/n^2$, then
$(p_m) = (2, 5/2, 25/9, 425/144, 442/144, \ldots)$,
which looks like it converges.
\medskip
\item{} b.

For the forward direction, assume that
$\sum _{n=1} ^\infty a_n$ converges to $a$.
Since the sequence of partial sums $(s_m)$ is monotonically increasing
and $1 + x \le 3^x$ for positive $x$,
$p_m = \prod _{n=1} ^m (1+a_n) \le \prod _{n=1} ^m 3^{a_n} =
3 ^{\sum _{n=1} ^m a_n} \le 3 ^a$, so $(p_m)$ is bounded.
Since $a_n \ge 0$ for all $n$ and $p_{m+1} = p_m (1 + a_n) \ge p_m$,
$(p_m)$ is increasing.
Therefore, by the monotone convergence theorem, $(p_m)$ converges, as desired.

For the backward direction, assume that
$\prod _{n=1} ^\infty (1+a_n)$ converges to $a$.
Since $a_n \ge 0$ for all $n$,
all terms of $\prod _{n=1} ^m (1 + a_n)$ are nonnegative,
and since the sequence of partial products $(p_m)$ is monotonically increasing,
$s_m = \sum _{n=1} ^m a_n \le \prod _{n=1} ^m (1+a_n) \le a$,
so $(s_m)$ is bounded.
Since $(s_m)$ is also increasing, by the monotone convergence theorem,
$(s_m)$ converges, as desired.
\bigskip
\item{2.6.2.} a. $a_n = (-1)^{n+1} (1/n)$

Clearly this sequence converges to $0$ (and is therefore Cauchy)
but is not monotone, since it alternates between positive and negative values.
\medskip
\item{} b. impossible

Since every Cauchy sequence is bounded,
there is a positive real number $M$ such that
$\abs{a_n} \le M$ for all $n$.
Therefore, for any subsequence $(a_{n_k})$ of $(a_n)$,
$\abs{a_{n_k}} \le M$ for all $k$, so $(a_{n_k})$ is bounded.
\medskip
\item{} c. impossible

Any divergent monotone sequence $(a_n)$ cannot be bounded, since
if it were bounded, by the monotone convergence theorem, it would converge.
If $(a_n)$ is increasing, then $(a_n)$ is bounded below by $a_1$
and is therefore not bounded above.
Since $(a_n)$ is not bounded above,
any positive real number $M$ is not an upper bound for $(a_n)$.
Therefore, there is some $N_1$ such that $a_{N_1} > M$.
Since $(a_n)$ is increasing, $a_n > M$ for all $n \ge N_1$.
Analogously, if $(a_n)$ is decreasing,
then for any positive real number $M$
there is some $N_2$ such that $a_n < -M$ for all $n \ge N_2$.
Therefore, eventually $\abs{a_n} > M$ for all positive real numbers $M$.
Let $(a_{n_k})$ be a subsequence of $(a_n)$.
Given a positive real number $M$, take $N$
such that $\abs{a_n} > M$ for all $n \ge N$.
Since $n_k \ge k$ for all $k$, $\abs{a_{n_k}} \ge M$ for all $k \ge N$.
Therefore, $(a_{n_k})$ is not bounded and is therefore not Cauchy.
\medskip
\item{} d. $a_n = n$ if $n$ is even and $a_n = 0$ if $n$ is odd

This sequence is not bounded, since given $M > 0$,
take $N$ such that $N > M$.
There is a even number $n \ge N$, so $a_n = n \ge N > M$.
The sequence $(a_{n_k}) = (0,0,\ldots)$ converges to $0$, so it is Cauchy.
\bigskip
\item{2.7.4.} a. $x_n = 1/n$, $y_n = (-1)^{n+1}$

The series $\sum _{n=1} ^\infty x_n y_n = \sum _{n=1} ^\infty (-1)^{n+1} (1/n)$,
which converges by the alternating series test.
\medskip
\item{} b. $x_n = (-1)^{n+1} (1/n)$, $y_n = (-1)^{n+1}$

The series $\sum _{n=1} ^\infty x_n$ converges by the alternating series test,
and $(y_n)$ is bounded since $\abs{y_n} \le 1$ for all $n$.
The series $\sum _{n=1} ^\infty x_n y_n = \sum _{n=1} ^\infty 1/n$
is the harmonic series, which diverges.
\medskip
\item{} c. impossible

Assume that $\sum x_n = a$ and $\sum (x_n + y_n) = b$.
Then by the algebraic limit theorem for series,
$\sum -x_n = -a$ and $\sum y_n = \sum (x_n + y_n - x_n) = b - a$,
so $\sum y_n$ must converge.
\medskip
\item{} d. $x_n = 1/n$ if $n$ is even, $x_n = 0$ if $n$ is odd

Let $s_m = \sum _{n=1} ^m x_n$.
Then $s_{2k} = 0 + 1/2 + 0 + 1/4 + \cdots + 0 + 1/(2k) =
(1/2) \sum _{n=1} ^k 1/n$ has partial sums
equal to half those of the harmonic series, which diverges.
\bigskip
\item{2.7.8.} a. true

We will show that $\sum a_n ^2$ is Cauchy.
Given $\epsilon > 0$, since $\sum \abs{a_n}$ is Cauchy,
we can take $N$ such that for all $n > m \ge N$,
$\sum _{k=m+1} ^n \abs{a_n} < \sqrt \epsilon$.
Squaring both sides of the inequality
and dropping the cross terms, which are nonnegative,
we have that $\sum _{k=m+1} ^n a_n ^2 < \epsilon$,
so $\sum a_n ^2$ is Cauchy, as desired.
\medskip
\item{} b. false

Let $a_n = b_n = (-1)^{n+1} / \sqrt n$.
Then $\sum a_n$ converges by the alternating series test,
while $(b_n) \to 0$.
The series $\sum a_n b_n = \sum 1/n$ is the harmonic series,
which diverges.
\medskip
\item{} c. true

We will prove this statement by contraposition,
that is, we will show that if $\sum n^2 a_n$ converges,
then $\sum a_n$ converges absolutely or diverges.
Assume that $\sum n^2 a_n$ converges.
Then the sequence $(n^2 a_n) \to 0$.
Therefore there is some natural number $N$ such that
$\abs{n^2 a_n} < 1$ for all $n \ge N$.
Therefore, since $n^2 \ge 0$, eventually $\abs{a_n} < 1/n^2$.
Since $\sum 1/n^2$ converges, by the comparison test,
$\sum \abs{a_n}$ converges as well,
so $\sum a_n$ converges absolutely, as desired.
\bigskip
\item{2.7.10.} a.

The product can be written in the form $\prod (1 + 1/2^n)$,
which converges by exercise 2.4.10.b since $\sum 1/n^2$ converges.
\medskip
\item{} b.

Each partial product is positive and smaller than the previous one,
so by the monotone convergence theorem, the sequence of partial products converges.

The product $\prod _{n=1} ^\infty (2k-1)/(2k)$ converges to $0$.
Let $p_m = \prod _{n=1} ^m (2k-1)/(2k)$ be the sequence of partial products.
Consider the product $\prod _{n=1} ^\infty (2k)/(2k+1)$
with the sequence of partial products $(q_m)$.
The sequence $(q_m)$ converges for the same reasons $(p_m)$ converges.
Notice that since $(2k + 1)(2k - 1) = 4k^2 - 1 < 4k^2 = (2k)(2k)$ for all $k$,
$(2k-1)/(2k) < (2k)/(2k+1)$ and $p_m < q_m$ for all $m$.
In addition, consider the product $\prod _{n=1} ^\infty n/(n+1)$,
which has the sequence $(r_m)$ of partial products.
Clearly, $(r_m) \to 0$, so the subsequence $(r_{2k}) \to 0$ as well.
Notice that $r_{2k} = p_k q_k$, so given $\epsilon > 0$,
there is some $K$ such that $p_k ^2 \le p_k q_k < \epsilon^2$
for all $k \ge K$.
Taking the square root of both sides, $p_k < \epsilon$, so $(p_k) \to 0$, as desired.
\medskip
\item{} c.

This product can be written in the form $\prod _{n=1} ^\infty (1 + 1/(4n^2 - 1))$.
Since $3n^2 > 1$ for natural numbers $n$, we have that $4n^2 - 1 > n^2$,
so $1/(4n^2 - 1) < 1/n^2$.
Since $\sum 1/n^2$ converges, by the comparison test, $\sum 1/(4n^2 - 1)$ converges as well.
By exercise 2.4.10.b, $\prod (1 + 1/(4n^2 - 1))$ converges, as desired.
\bye
