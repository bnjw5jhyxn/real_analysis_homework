\def\real{{\bf R}}
\def\rational{{\bf Q}}
\def\natural{{\bf N}}
\def\abs#1{\vert#1\vert}
\centerline{Steve Hsu\hfill homework 1}
\bigskip
\item{1.2.6.} a.

If $a$ and $b$ are both nonnegative,
then the inequality simplifies to $a + b \le a + b$, which is clearly true.

If they are both negative, then it simplifies to $-(a + b) \le -a - b$,
which is also clearly true.
\medskip
\item{} b.

Lemma 1. $x \le \abs x$ for any real number $x$.

Proof. We have two cases: first, $x$ is nonnegative;
and second, $x$ is negative.

In the first case, the inequality simplifies to $x \le x$,
which is clearly true.

In the second case, it simplifies to $x \le -x$.
Since $x$ is negative and $-x$ is positive,
this inequality is also true.
\medskip
Lemma 2. $\abs a \abs b = \abs{ab}$ for any real numbers $a$ and $b$.

Proof. We have two cases: first, $a$ and $b$ have the same sign;
and second, $a$ and $b$ have different signs.

In the first case, the equation simplifies to $ab = ab$,
which is clearly true.

In the second case, it simplifies to $-ab = -ab$, which is also clearly true.
\medskip
Lemma 3. If $a^2 \le b^2$,
then $\abs a \le \abs b$ for any real numbers $a$ and $b$.

Proof by contraposition. If $\abs a > \abs b$,
then $a^2 > b^2$ for any real numbers $a$ and $b$.

Assume that $\abs a > \abs b$.
We want to show that $a^2 > b^2$.

Multiplying both sides by $\abs a$ gives $a^2 > \abs{ab}$,
while multiplying by $\abs b$ gives $\abs{ab} > b^2$,
so $a^2 > b^2$, as desired.
\medskip
We want to show that $\abs{a + b} \le \abs a + \abs b$
for any real numbers $a$ and $b$.
By lemma 1, $2ab \le \abs{2ab}$.
Adding $a^2 + b^2$ to both sides,
$a^2 + 2ab + b^2 \le a^2 + \abs{2ab} + b^2$.
Since $x^2 = \abs x^2$ for any real number $x$ by lemma 2,
the right hand side is equal to ${\abs a}^2 + 2\abs a\abs b + \abs b^2$.
Factoring both sides, $(a + b)^2 \le (\abs a + \abs b)^2$.
By lemma 3, we can simplify the inequality to
$\abs{a + b} \le \abs{\abs a + \abs b}$.
Since $\abs a + \abs b$ is nonnegative,
the right hand side simplifies to $\abs a + \abs b$.
We now have $\abs{a + b} \le \abs a + \abs b$, as desired.
\medskip
\item{} c.

By the triangle inequality,
$\abs{a - c} + \abs{c - d} \ge \abs{a - c + c - d} = \abs{a - d}$.
Adding $\abs{d - b}$ to both sides,
$\abs{a - c} + \abs{c - d} + \abs{d - b} \ge \abs{a - d} + \abs{d - b}$.
Applying the triangle inequality to the right hand side,
$\abs{a - d} + \abs{d - b} \ge \abs{a - b}$, as desired.
\medskip
\item{} d.

We have two cases: first, $\abs a \ge \abs b$; and second, $\abs b > \abs a$.

In the first case, by the triangle inequality,
$\abs a = \abs{a - b + b} \le \abs{a - b} + \abs{b}$.
Subtracting $\abs b$ from both sides,
$\abs a - \abs b \le \abs{a - b}$.
Since $\abs a \ge \abs b$,
then $\abs{\abs a - \abs b} = \abs a - \abs b \le \abs{a - b}$, as desired.

In the second case, observe that
$\abs{\abs a - \abs b} = \abs{\abs b - \abs a}$.
By a similar argument,
$\abs{\abs b - \abs a} = \abs b - \abs a \le \abs{b - a} = \abs{a - b}$,
as desired.
\bigskip
\item{1.2.11.} a.

There are real numbers satisfying $a < b$ such that
for all natural numbers $n$, $a + 1/n \ge b$.

The claim is true.
\medskip
\item{} b.

For all positive real numbers $x$, $x \ge 1/n$ for some natural number $n$.

The negation is true.
\medskip
\item{} c.

There are two distinct real numbers such that
all numbers between them are irrational.

The claim is true.
\bigskip
\item{1.3.10.} a.

Notice that all elements of $B$ are upper bounds for $A$.
Since $B$ is nonempty, $A$ has an upper bound and $\sup A$ exists.
By definition, $\sup A \le b$ for any $b$ in $B$
and $\sup A \ge a$ for any $a$ in $A$.
\medskip
\item{} b.

Lemma. For real numbers $a$ and $b$,
if $a + \epsilon \ge b$ for any positive real number $\epsilon$,
then $a \ge b$.

Proof by contraposition.
Let $a$ and $b$ be real numbers.
If $a < b$, then there is a positive real number $\epsilon$
such that $a + \epsilon < b$.

Assume that $a < b$.
We want to show that there is a positive real number $\epsilon$
such that $a + \epsilon < b$.

Since $b > a$, $(b - a) / 2$ is a positive real number.
Setting $\epsilon = (b - a) / 2$,
$a + \epsilon = a + b/2 - a/2 = a/2 + b/2 = b - (b - a)/2$,
which is clearly less than b, as desired.
\medskip
Let $B$ be the set of upper bounds for $E$.
Let $A$ be the complement of $B$ in $\real$.
$A$ and $B$ are disjoint and have union $\real$
by the definition of complement.
$B$ is nonempty since $E$ is bounded above.
Since $E$ is nonempty, there is a real number $e$ in $E$.
$e - 1$ is clearly not an upper bound for $E$, so $A$ is nonempty.
By the definition of upper bound,
all elements of $A$ are smaller than all elements of $B$.
By the cut property, there is some real number $c$ such that
$a \le c$ for all $a \in A$ and $c \le b$ for all $b \in B$.
We claim that $c = \sup E$.
We must first show that $c$ is an upper bound for $E$.
In other words, $c \ge e$ for any $e \in E$.
Let $e$ be an arbitrary element of $E$.
For any positive real number $\epsilon$,
$e - \epsilon$ is not an upper bound for $E$,
so $e - \epsilon \in A$ and $c \ge e - \epsilon$.
In other words, $c + \epsilon \ge e$.
By the lemma, this implies that $c \ge e$,
so $c$ is an upper bound for $E$.
By the definition of the cut property,
$c$ is the least upper bound for $E$, as desired.
\medskip
\item{} c.

Let $B = \{x \in \rational : x > 0, x^2 > 2\}$
be the set of positive rational numbers whose squares are greater than $2$.
Let $A$ be the complement of $B$ in $\rational$.
$A$ and $B$ are disjoint, nonempty sets whose union is $\rational$
satisfying $a < b$ for all $a \in A$ and $b \in B$,
but there is no rational number satisfying the cut property.
\bigskip
\item{1.4.8.} a.
$A = \{1 - ({1 \over 2})^n : n \in \natural\},
B = \{1 - ({1 \over 3})^n : n \in \natural\}$
\medskip
\item{} b.
$J_k = (-{1 \over n}, {1 \over n})$
is the open interval from $-{1 \over n}$ to $1 \over n$
\medskip
\item{} c.
$L_k = [k, \infty)$
\medskip
\item{} d.
impossible

Let $M_k = \bigcap_{n=1}^k I_n$.
$M_1, M_2, M_3, \ldots$ is a sequence of nested intervals.
Clearly, $\bigcap_{n=1}^\infty M_n = \bigcap_{n=1}^\infty I_n$.
By the nested interval property, this intersection is nonempty.
\bye
