\centerline{Steve Hsu\hfill workshop 9}
\item{b.}

In order to show that $f$ is differentiable at $0$, we need to show
the existence of $f'(0) = \lim _{x \to 0} (f(x) - f(0)) / (x - 0)$.
Since $f$ is continuous at $0$ and differentiable close to $0$,
we can apply L'Hospital's rule to obtain
$f'(0) = \lim _{x \to 0} (f'(x) - 0) / (1) = \lim _{x \to 0} f'(x)$.
Since $f'(x) = 0$ for $x < 0$ and $f'(x) = 2 x^{-3} e^{-x^{-2}}$ for $x > 0$,
we have that
$\lim _{x \to 0^+} f'(x) = \lim _{x \to 0^+} x^{-3} e^{-x^{-2}}
= \lim _{x \to 0^+} 2 / (x^3 e^{x^{-2}})$,
which is clearly $0$.
Therefore, $\lim _{x \to 0} f'(x) = 0$, so $f'(0) = 0$
and $f'$ exists and is continuous at $0$.
Since $f'$ is continuous everywhere else,
$f$ is differentiable and $f'$ is continuous, as desired.
\bigskip
\item{} c.

We want to show the existence of
$f''(0) = \lim _{x \to 0} (f'(x) - f'(0)) / (x - 0)$.
Since $f'$ is continuous at $0$ and differentiable close to $0$,
we can apply L'Hospital's rule to obtain
$f''(0) = \lim _{x \to 0} (f''(x) - 0) / (1) = \lim _{x \to 0} f''(x)$.
Since $f''(x) = -6 x^{-4} e^{-x^{-2}} + 4 x^{-6} e^{-x^{-2}}$ for $x > 0$,
we have that $\lim _{x \to 0^+} f''(x) = 0$
for the same reasons as in part (a).
Therefore, $f''(0) = \lim _{x \to 0} f''(x) = 0$,
so $f''$ is continuous at $0$.
By the same argument as in part (a),
we have that $f'$ is differentiable and $f''$ is continuous.
\bye
