\def\abs#1{\vert#1\vert}
\centerline{Steve Hsu\hfill workshop 8}
\item{1.}

Given $\epsilon > 0$, we want to find $\delta > 0$ such that
for all $x, y \ge 0$, if $\abs{x - y} < \delta$, it follows that
$\abs{g(x) - g(y)} < \epsilon$.

Since $\lim _{x \to +\infty} g(x)$ exists and is finite,
we can take $L = \lim _{x \to +\infty} g(x)$.
By definition, there exists a real number $M$ such that
whenever $x \ge M$, it follows that $\abs{g(x) - L} < \epsilon/2$.
Therefore, if $x, y \ge M$, then $\abs{x - y} = \abs{x - M + M - y} \le
\abs{x - M} + \abs{y - M} < \epsilon/2 + \epsilon/2 = \epsilon$.
Since $g$ is continuous on $[0,M+1]$, by uniform continuity on compact sets,
it follows that $g$ is uniformly continuous on $[0,M+1]$.
Therefore, there exists $\delta _1$ such that for all $x, y \in [0,M+1]$,
if $\abs{x - y} < \delta _1$, it follows that $\abs{x - y} < \epsilon$.
Take $\delta = \min \{1, \delta _1\}$.
Notice that for all $x, y \ge 0$, if $\abs{x - y} < \delta$,
we cannot have one in $[0,M)$ and the other in $(M+1,+\infty)$
since $\delta \le 1$.
Therefore, either $x, y \in [0,M+1]$ or $x, y \ge M$.
In either case, by the reasoning above, $\abs{x - y} < \epsilon$, as desired.
\bye
