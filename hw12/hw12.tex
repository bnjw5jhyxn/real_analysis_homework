\def\abs#1{\vert#1\vert}
\centerline{Steve Hsu\hfill homework 12}
\item{5.2.2.} a. $f(x) = g(x) = \abs x$ for all real numbers $x$

We know that $\abs x$ is not differentiable at $0$,
but $\abs x^2 = x^2$ is differentiable at $0$.
\medskip
item{} b. $f(x) = \abs x$ and $g(x) = 0$ for all real numbers $x$

Notice that $(fg)(x) = 0$ for all real numbers $x$.
The desired property follows.
\medskip
\item{} c. not possible

Suppose for contradiction that there are functions $f$ and $g$
such that $f$ is not differentiable at $0$,
but $g$ and $f+g$ are.
Then $g'(0)$ and $(f+g)'(0)$ both exist.
By the algebraic differentiability theorem, we have that
$(f+g)'(0) - g'(0)$ exists and is equal to $f'(0)$.
\medskip
\item{} d. $$f(x) = \cases{
x^2 &, $x$ is rational\cr
0 &, $x$ is irrational\cr
}$$

Notice that $f$ is discontinuous at all points except $0$
and is therefore not differentiable at any point other than $0$.
Additionally, $f(x) \le x^2$ for all real numbers $x$.
We want to show that $\lim _{x \to 0} (f(x) - f(0)) / (x - 0)$ exists.
We claim that $\lim _{x \to 0} (f(x) - f(0)) / (x - 0) = 0$.
Given $\epsilon > 0$, take $\delta = \epsilon$.
If $\abs{x - 0} < \delta$, then we have that
$\abs{(f(x) - f(0)) / (x - 0) - 0} = \abs{(f(x) - 0) / x} \le \abs{x^2 / x} =
\abs x < \delta = \epsilon$, as desired.
\bigskip
\item{5.2.5.} a. If $a > 0$, then $f_a$ is continuous at $0$.

If $a > 0$, then $\lim _{x \to 0^+} f(x) = 0$.
If $a = 0$, then $\lim _{x \to 0^+} f(x) = 1$.
If $a < 0$, then $\lim _{x \to 0^+} f(x) = +\infty$.
Since $f(0) = \lim _{x \to 0^-} f(x) = 0$,
$f$ is continuous at $0$ if and only if $a > 0$.
\medskip
\item{} b. If $a > 1$, then $f_a$ is differentiable at $0$.

If $a > 1$, then $\lim _{x \to 0^+} (f(x) - f(0)) / (x - 0) =
\lim _{x \to 0^+} (x^a - 0^a) / x = \lim _{x \to 0^+} x^{a - 1}$.
\hfil\break
Since $\lim _{x \to 0^-} (f(x) - f(0)) / (x - 0) =
\lim _{x \to 0} (0 - 0) / x = 0$,
we have that $\lim _{x \to 0}$ exists if and only if $a - 1 > 0$,
that is, if $a > 1$.

Since $f_a'(x) = ax^{a - 1}$ when $x > 0$,
and $a0^{a - 1} = 0$ when $a > 1$,
we have that $f_a'$ is continuous at $0$ when it exists.
\medskip
\item{} c. If $a > 2$, then $f_a$ is twice differentiable at $0$.

By part (b), we have that $f_a' = af_{a-1}$,
which is differentiable if and only if $a - 1 > 1$, that is,
if $a > 2$.
\bigskip
\item{5.2.9.} a. true

Assume that $f'$ is not constant.
Then there exist $x$ and $y$ such that $f'(x) \ne f'(y)$.
Since $f'$ has the intermediate value property,
and by the density of the irrational numbers in the real numbers,
there is an irrational number between $f'(x)$ and $f'(y)$
and $f'$ attains this value.
\medskip
\item{} b. false: take $f(x) = {1 \over 2} x + x^2 \sin (1/x)$

Notice that $f'(0) = 1/2$ and that
$f'(x) = 1/2 + 2x \sin (1/x) - \sin (1/x)$,
which oscillates between $-1/2$ and $3/2$ when $x$ is close to $0$.
\medskip
\item{} c. true

Assume for contradiction that there is a function $f$
differentiable on an interval containing $0$ and a real number $L$
such that $\lim _{x \to 0} f'(x) = L$, but $f'(0) \ne L$.
Then there exists $\delta > 0$ such that if $0 < \abs{x - 0} < \delta$,
then $\abs{f'(x) - L} < {1 \over 2} \abs{L - f'(0)}$.
Notice that ${1 \over 4} L + {3 \over 4} f'(0)$ is between $f'(0)$ and $L$,
but outside the $\abs{L - f'(0)}$-neighborhood of $L$.
This is a contradiction because
$f'$ is a derivative, but does not satisfy the intermediate value property.
\bigskip
\item{5.3.3.} a.

Since $h$ is differentiable on $[0,3]$,
it follows that $h$ is continuous on $[0,3]$.
By the algebraic continuity theorem,
$g(x) = h(x) - x$ is also continuous on $[0,3]$.
Notice that $g(0) = h(0) - 0 = 1 - 0 = 1$
and that $g(3) = h(3) - 3 = 2 - 3 = -1$.
By the intermediate value theorem,
there is a real number $c \in [0,3]$ such that $g(c) = h(c) - c = 0$.
It follows that $h(c) = c$, as desired.
\goodbreak
\item{} b.

By the mean value theorem,
there is a point $c \in [0,3]$ such that
$h'(c) = (h(3) - h(0)) / (3 - 0) = (2 - 1) / 3 = 1/3$,
as desired.
\medskip
\item{} c.

By the mean value theorem, there is a point $c_1 \in [0,1]$ such that
$h'(c_1) = (h(1) - h(0)) / (1 - 0) = (2 - 1) / 1 = 1$.
There is also a point $c_2 \in [1,3]$ such that
$h'(c_2) = (h(3) - h(1)) / (3 - 1) = (2 - 2) / 2 = 0$.
Since $h'$ satisfies the intermediate value property,
it follows that there is a point $c \in [c_1, c_2]$ such that
$h'(c) = 1/4$, as desired.
\bigskip
\item{5.3.8.}

We want to show the existence of
$f'(0) = \lim _{x \to 0} (f(x) - f(0)) / (x - 0)$.
Since $f$ is continuous, we have that $\lim _{x \to 0} f(x) = f(0)$, that is,
$\lim _{x \to 0} f(x) - f(0) = 0$.
Additionally, $\lim _{x \to 0} x - 0 = 0$.
Since $f$ is differentiable close to $0$, we can apply L'Hospital's Rule,
obtaining $f'(0) = \lim _{x \to 0} f'(x) / 1 = L$, as desired.
\bye
