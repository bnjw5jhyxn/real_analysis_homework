\def\abs#1{\vert#1\vert}
\def\real{{\bf R}}
\centerline{Steve Hsu\hfill homework 9}
\item{4.2.9.} a.

Given $M > 0$, take $\delta = 1 / \sqrt M$.
If $\abs{x - 0} = \abs x < \delta$,
then $1 / x^2 = 1 / \abs x^2 > 1 / (1 / \sqrt M)^2 = \sqrt M^2 = M$,
as desired.
\medskip
\item{} b.

Let $f : A \to \real$ be such that
for all $x \in R$, $(x, +\infty) \cap A$ is nonempty.
We say that $\lim _{x \to +\infty} f(x) = L$ if
for every $\epsilon > 0$, there exists $M > 0$ such that
whenever $x > M$, it follows that $\abs{f(x) - L} < \epsilon$.

Given $\epsilon > 0$, take $M = 1 / \epsilon$.
When $x > M$, $\abs{1/x - 0} = \abs{1/x} =
1/x < 1 / (1 / \epsilon) = \epsilon$, as desired.
\medskip
\item{} c.

Let $f : A \to \real$ be such that
for all $x \in R$, $(x, +\infty) \cap A$ is nonempty.
We say that $\lim _{x \to +\infty} f(x) = +\infty$ if
for every $M > 0$, there exists $N > 0$ such that
whenever $x > N$, it follows that $f(x) > M$.

For example, $\lim _{x \to +\infty} x = +\infty$.
Given $M > 0$, take $N = M$.
If $x > N$, then $x > N = M$, as desired.
\bigskip
\item{4.3.9.}

We want to show that $K$ contains all of its limit points.
Let $x$ be a limit point of $K$.
Then there is a sequence $(x_n) \subset K$ that converges to $x$.
By the definition of $K$, we have that
$h(x_n) = 0$ for all natural numbers $n$.
Since $h$ is continuous on $\real$ and $x$ is a limit point of $K$,
by the sequential criterion for functional limits,
we have that $h(x) = \lim h(x_n) = 0$,
so $x \in K$, as desired.
\bigskip
\item{4.3.12.}

We want to show that $g$ is continuous at $c$ for all real numbers $c$.
Let $c$ be a real number.
Given $\epsilon > 0$, take $\delta = \epsilon$.
Notice that $\abs{g(x) - g(c)} = \abs{\inf \{\abs{x - a} : a \in F\} - g(c)} =
\abs{\inf \{\abs{x - c + c - a} : a \in F\} - g(c)} \le
\abs{\inf \{\abs{x - c} + \abs{c - a} : a \in F\} - g(c)} =
\abs{\abs{x - c} + \inf \{\abs{c - a} : a \in F\} - g(c)} =
\abs{\abs{x - c} + g(c) - g(c)} = \abs{\abs{x - c}} = \abs{x - c} <
\delta = \epsilon$, so $g$ is continuous at $c$, as desired.

We want to show that if $x \notin F$, then $g(x) \ne 0$.
Let $x$ be a real number not in $F$.
Since $F$ is closed, we have that $x$ is not a limit point of $F$.
Therefore, there exists $\epsilon > 0$ such that
$\abs{x - a} \ge \epsilon$ for all $a \in F$.
Therefore, we have that $\epsilon$ is a lower bound for
$\{\abs{x - a} : a \in F\}$, so $g(x) \ge \epsilon > 0$, as desired.
\bye
